\documentclass{book}
\usepackage[brazilian]{babel}
\usepackage[utf8]{inputenc}


\author{Emílio Wuerges}
\title{Plano de Aula - Computação Distribuída}
\begin{document}



\chapter{Revisão de Redes de Computadores}


\section{Ethernet/Camada física + enlace}

Unidade transmitida se chama de Frame.

Responsabilidades:
\begin{itemize}
\item Prevenir colizões,
\item Dectectar colizões (É importante lembrar que em protocolos sem fio 
costuma ser impraticável detectar colizões),
\item Recentemente: Detectar erros nas mensagens (checksum),
\item Tratar colizões,
\item Retrasmitir ou
\item Falhar.
\end{itemize}

\section{IP/Camada de Rede}

Unidade transmitida se chama de Pacote.

Responsabilidades:

\begin{itemize}
\item Transferir dados entre interfaces de rede em diferentes enlaces,
\item Encontrar rotas,
\item Evitar congestionamento da rede (derrubando pacotes), e
\item Detectar erros nas mensagens (checksum).
\end{itemize}

\section{UDP/Camada de Transporte}

Unidade transmitida se chama de Mensagem.

Responsabilidades:
\begin{itemize}
\item Disponibilizar os pacotes IP para as aplicações.
\end{itemize}

\section{UDP/Camada de Transporte}

Não se trata mais de unidades, mas de streams de mensagem.

Responsabilidades:
\begin{itemize}
\item Mesmas do UDP,
\item Controle de Banda,
\item Garantia de entrega,
\item Retransmissão,
\item Ordenação de mensagens, e
\item Controle de congestionamento (com ajuda da camada de rede).
\end{itemize}

\chapter{Comunicação Cliente/Servidor}

\section{URI/URL}
\begin{itemize}
\item Identifica/endereçar tudo.
\item No nosso caso: Documentos e procedimentos remotos.
\end{itemize}

\section{HTTP}

\begin{itemize}
\item HTTP é um protocolo no nível de aplicação,
\item Usa TCP,
\item Criado para gerenciar documentos pela internet.
\end{itemize}

Possui vários tipos de métodos (CUIDADO, método aqui não significa método da orientação a objeto):

\begin{itemize}
\item GET 
É o mais comum. Usado para ler documentos.

\item HEAD
Quase o mesmo que o get, mas apenas lê o cabeçalho dos documentos.

\item PUT
Usado para modificar um documento.

\item DELETE
Usado para eliminar documentos.

\item POST
Usado para criar documentos novos.

\end{itemize}

\section{XML}

Existe. 

É uma maneira type-safe (que é muito bom) de se representar dados estruturados.

Entretanto,
já que a maioria das ferramentas usadas não é type-safe, 
não faz muito sentido usar. 
Afinal, a força de uma corrente é seu elo mais fraco.

\section{JSON}

É a notação usada para escrever os dados na linguagem JavaScript.
É muito usada para substituir o XML devido a praticidade de implementação.

Maneira recomendada de se representar os argumentos e o retorno das chamadas ReST.

\section{ReST}

Essencialmente é uma enjambração que usa URIs e o HTTP para oferecer (e gerenciar) 
recursos (no nosso caso, objetos e métodos - esse sim é da orientação a objetos) na internet.

Atualmente é a maneira mais comum de se oferecer uma API web.

No final das contas, é uma maneira elegante de se fazer 
chamadas de métodos independente de ambiente de desenvolvimento
e linguagem de programação usando apenas padrões abertos.

Recursos são representados por uma URI base: http://minhaurlexemplo.com/lista\_de\_objetos


É extremamente comum ver esquemas de URIs desta maneira representando recursos:

\begin{itemize}
\item GET + http://minhaurlexemplo.com/lista\_de\_objetos/nome\_do\_obj : Retorna um JSON com o objeto.
\item PUT + http://minhaurlexemplo.com/lista\_de\_objetos/nome\_do\_obj + JSON: Altera os campos do objeto que coincidirem com o JSON.
\item POST + http://minhaurlexemplo.com/lista\_de\_objetos/create + JSON: Cria um objeto com os dados do JSON.
\item DELETE + http://minhaurlexemplo.com/lista\_de\_objetos/nome\_do\_obj: Apaga o objeto.
\end{itemize}

Métodos podem ser representados por mais itens no caminho.

\begin{itemize}
\item http://minhaurlexemplo.com/objeto/metodo/parametro1/parametro2/etc
\end{itemize}

MUITA ATENÇÃO para o método HTTP escolhido!
Espera-se que GET, PUT e DELETE sejam idempotentes. Ações não idempotentes devem usar POST.

Violar esta condição pode causar comportamento demoníaco devido a otimizações de cache e proxies.

\chapter{Comunicação em Grupo}

\section{Basic Multicast}
Suponha que um participante possua um conjunto de pares com qual precisa se comunicar.

Quais os problema do algoritmo da Figura \ref{alg:basic}, no caso de desconexão de algum dos peers?

\begin{figure}
\begin{verbatim}
def basic_multicast(m, group):
  for p in group:
    send(p, m)

def basic_deliver(m):
  entrega mensagem para aplicação
\end{verbatim}
  \caption{Algortimo Basic Multicast}
  \label{alg:basic}
\end{figure}

O primeiro problema já ocorre se o \texttt{send} for bloqueante.
Caso um dos peers desconectar, o send será bloqueado indefinidamente.

E se caso o \texttt{send} não seja bloqueante. O algoritmo irá progredir.

Entretanto, e se o participante que iniciou os envios falhar, de qualquer maneira
apenas uma parte dos peers receberá a mensagem.

\section{Reliable Multicast}

\begin{figure}
\begin{verbatim}

def on_initialize():
  r = {}

# envio no processo p
def reliable_multicast(m, group):
  basic_multicast(m, group)
  
def basic_deliver(m):
  entrega mensagem para aplicação

# recepção no processo q
def reliable_deliver(m):
  if not m in r:
    r.add(m)
    if not p is q:
      basic_multicast(m, group)
    basic_deliver(m)

\end{verbatim}
  \caption{Reliable Multicast}
  \label{alg:reliable}
\end{figure}

O algoritmo da Figura \ref{alg:reliable} resolve o problema.
O mecanismo de envio de mensagens é inalterado.
Entretanto, ao receber, o processo receptor precisa verificar
se a mensagem que ele recebeu já foi recebida. Se não foi, ele envia
novamente todas as mensagens para todos do grupo.

Como cada grupo retransmitirá todas as mensagens uma vez, 
e cada retransmissão manda todas as mensagens para o grupo todo,
o custo é $O(n^2)$.

Entretanto, a vantagem do algoritmo é que ele pode falhar em cada momento que as seguintes
propriedades serão respeitadas:

Validade: Se um processo efetua o multicast, eventualmente ele entrega a mensagem
para a aplicação.

Integridade: Se um processo correto entrega a mensagem, todos os processos entregam.


\section{Multicast Síncrono}

\begin{figure}
\begin{verbatim}

def on_initialize():
  lm = {} # conjunto local de mensagens

def poll_for_messages():
  for p in group:
    ms = get_messages(p)
    for m in ms:
      if not m in lm:
        basic_deliver(m)
    lm.union_with(ms)

a cada intervalo de tempo: poll_for_messages()
\end{verbatim}
\caption{Algoritmo para recepção de mensagens síncrono}
\label{alg:sinc}
\end{figure}

O algoritmo da Figura \ref{alg:sinc} apresenta uma solução síncrona para o problema.
Ele apresenta também as propriedades de Validade e Integridade, entretanto, ele
depende de um tempo mínimo, em que as mensagens precisam ser propagadas para os participates do grupo.

O custo deste algoritmo é $O(n \times m)$, onde $n$ é o número de membros do grupo e $m$ é o número de 
mensagens enviadas durante a rodada.

Esta abordagem, apesar de ser síncrona, é a sugerida para os trabalhos da disciplina
pois sua implementação usando ReST é trivial.

\chapter{Relógios Lógicos}


\begin{figure}
\begin{verbatim}
def on_initialize():
  l = 0

def on_event():
  l = l + 1

def clock_send(d, m):
  send(d, [m, l]) # agrega o relógio local a mensagem enviada

def on_clock_receive([m, l_r]):
  if l_r > l:
    l = l_r
  on_event()
  basic_deliver(m)

\end{verbatim}
  \caption{Relógios de Lamport}
  \label{alg:lamport}
\end{figure}


\begin{figure}
\begin{verbatim}

def on_initialize(): #inicialização no processo p
  l = { p: 0 }

def on_event():
  l[p] = l[p] + 1

def vector_send(d, m):
  send(d, [m, l]) # agrega o relógio local a mensagem enviada

def on_vector_receive([m, l_r]):
  for (q, t) in l_r:
    if not (q in p and p[q] > t):
      p[q] = t
  on_event()
  basic_deliver(m)

\end{verbatim}
  \caption{Relógios Vetor}
  \label{alg:vector}
\end{figure}

\end{document}




